\documentclass{beamer}


\usepackage[spanish]{babel}
\usepackage[T1]{fontenc}
\usepackage[utf8]{inputenc}

\usepackage{graphics}

\usepackage{fancyvrb} %fancy verbatim
\usepackage{color}

\graphicspath{{img/}}

\usetheme{Madrid}
\usecolortheme{beaver}
\setbeamercovered{transparent}

% \usepackage{beamerthemesplit} // Activate for custom appearance


\title{Modelación Basada en Agentes}
\author{Dr. Felipe Contreras}
\date{\today}

\begin{document}

\frame{\titlepage}

\section[Outline]{}
\frame{\tableofcontents}

\section{Antecedentes}

\subsection{Sistemas Complejos}
\frame{\alert{Sistemas Complejos}}
  \setbeamercovered{transparent}
\begin{frame}[t]
  \frametitle{Características}
  \begin{itemize}[<+-| alert@+>]
  \item Sistema: Conjunto de elementos o partes conectadas entre sí, que llevan acabo cierta función
  \item Complejo $\neq$ Complicado
  \item Presentan auto-organización
  \item Exhiben propiedades emergentes (``el todo es más que la suma de sus partes'')
  \begin{enumerate}[<+-| alert@+>]
  \item Muchos elementos
  \item Las interacciones son dinámicas
  \item Elementos influyen y son influidos por los demás
  \item Las interacciones son no lineales (pequeñas ``causas'', pueden tener ``efectos'' grandes)
  \item Las interacciones son recursivas
  \item Son abiertos % interacturan con otros sistemas
  \item Operan lejos del equilibrio
  \item Tienen una historia
  \item Los elementos actúan con información local
  \end{enumerate}
  \end{itemize}
  \end{frame}

\subsection{Mapeos Discretos}
\frame{\alert{Mapeos Discretos}}
\frame{
  \frametitle{Mapeos discretos}
  \begin{itemize}[<+->]
  \item $y = f(x)$
  \item $x_{1} = f(x_{0})$
  \item $x_{2} = f(x_{1}), x_{3}=f(x_{2}), x_{4} = f(x_{3}), ...$
  \end{itemize}
}
\frame{\frametitle{Representación gráfica}
\begin{block}{}
\begin{center}
\includegraphics<+>[height=.5\textheight]{mapeo1}
\includegraphics<+>[height=.5\textheight]{mapeo2}
\only<3>{ Programa ``logistica''}
\includegraphics<+>[width=.8\textwidth]{mapeo3}
\end{center}
\end{block}
}
\frame
{
  \frametitle{Características}

  \begin{itemize}[<+->]
  \item Converge (a un punto)
  \item No converge: tiene ciclo límite
  \item No converge: órbita densa
  \end{itemize}
}

\subsection{Autómatas Celulares}
\frame{\alert{Autómatas Celulares}}

\begin{frame}[t]
  \frametitle{Definición (genérica)}
  \begin{itemize}[<+->]
  \item Un AC consiste de autómatas (llamados también celdas o sitios) idénticos, dispuestos uniformemente en los puntos de una látice $D$-dimensional de un espacio discreto. Normalmente D=$1, 2,$ o $3$
  \item Cada autómata es una variable dinámica y su cambio temporal esta dado por la expresión:
  $$s_{t+1}(x) = F (s_{t}(x + x_{0}), s_{t}(x + x_{1}),\ldots , s_{t}(x + x_{n-1}))$$
  \item $s_{t}(x)$ es el estado de un autómata localizado en $x$ en el tiempo $t$
  \item $F$ es la función de transición de estado
  \item y $N=\{x_{0}, x_{1}, \ldots, x_{n-1}\}$ es la \textit{vecindad}
  \item por lo general se aplica la misma $F$ y la misma vecindad uniformemente a todas las posiciones espaciales
  \end{itemize}
\end{frame}

\begin{frame}[t]
  \frametitle{Definición (genérica)}
  \begin{itemize}[<+->]
  \item Fueron inicialmente desarrolladas por John von Newmann y su colaborador Stanislaw Ulam 
  \item Constituyen una forma de describir dinámicas espacio-temporales altamente no lineales de una manera simple y concisa
  \item Se utilizan para diversos campos como la dinámica molecular, hidrodinámica, propiedades físicas de materiales, procesos químicos de reacción-difusión, crecimiento y morfogénesis de organismos vivos, etc. [Sayama p.185 y ss]
  \end{itemize}
\end{frame}

\begin{frame}[t]
  \frametitle{AC 1D: Definición práctica}
  \begin{columns}[t]
  \column{.5\textwidth}
  \begin{block}{}
	\begin{itemize}[<+->]
		\item Vocabulario $\sigma$ de $n$ símbolos
		\item Organización de $m$ de estos símbolos en un estado inicial $E_{0}$
		\item Tamaño de vecindad o radio $\rho$
		\item Condiciones en la frontera (cíclica, terminación, valor único)
		\item Regla de evolución (función de mapeo)
	\end{itemize}
  \end{block}
   \column{.5\textwidth}
  \begin{center}
	\only<1> {$\sigma = \{0, 1\}$, $n=2$}
	\includegraphics<2>[width=.9\textwidth]{automata1}
	\only<3> {$\rho=3$}
	\includegraphics<4>[width=.9\textwidth]{automata2}
	\includegraphics<5>[height=.4\textheight]{automata3}
  \end{center}
  \end{columns}
\end{frame}

\begin{frame}[t]
  \frametitle{Regla de evolución}
  \includegraphics[width=.9\textwidth]{automata4}
\end{frame}

\begin{frame}[t]
  \frametitle{Aplicación de la regla}
  \includegraphics<+>[width=.9\textwidth]{automata51}
  \includegraphics<+>[width=.9\textwidth]{automata52}
  \includegraphics<+>[width=.9\textwidth]{automata53}
  \includegraphics<+>[width=.9\textwidth]{automata54}
  \includegraphics<+>[width=.9\textwidth]{automata55}
  \includegraphics<+>[width=.9\textwidth]{automata56}
  \includegraphics<+>[width=.9\textwidth]{automata57}
  \includegraphics<+>[width=.9\textwidth]{automata58}
  \includegraphics<+>[width=.9\textwidth]{automata59}
  \only<+>{?`cómo quedan las demás?, ?`cómo funciona la condición de frontera cíclica?}
\end{frame}

\begin{frame}[t]
  \frametitle{Ejemplos (Programa ``Autómatas Celulares'')}
  \begin{center}
  \only<+> {Regla 90, $E_{0}$=``central'' }
  \only<+> {Regla 94, $E_{0}$=``01110000000000001111100001111'' }
  \only<+> {Regla 135, $E_{0}$=``azar'' }
  \end{center}
  \begin{center}
  \includegraphics<1>[height=.6\textheight]{ac11}
  \includegraphics<2>[height=.6\textheight]{ac12}
  \includegraphics<3>[height=.6\textheight]{ac13}
  \end{center}
\end{frame}

\begin{frame}[t]
\frametitle{Clasificación de Wolfram}
\begin{itemize}[<+-| alert@+>]
	\item Uniforme
	\item Cíclico
	\item Aleatorio
	\item Complejo
\end{itemize}
\end{frame}

\begin{frame}[t]
\frametitle{Autómatas en 2D: Regla de la mayoría}
\begin{itemize}[<+->]
	\item La vida o muerte de la celda central está dictada por el valor de la mayoría de las celdas de su vecindad de Moore
\end{itemize}
\begin{center}
	\includegraphics<1>[width=.3\textwidth]{moore}
	\only<2> {Programa ``Manchas''}
	
	\includegraphics<2>[width=.5\textwidth]{manchas}
\end{center}
\end{frame}

\begin{frame}[t]
  \frametitle{Autómatas en 2D: El juego de la vida}
  \begin{itemize}[<+->]
  \item {El conjunto de símbolos es $\sigma=\{0,1\}$, significando 0=``muerta'', 1=``viva''}
  \item {$E_{0}$, y todos los demás estados, están dispuestos en una parrilla 2D de celdas}
  \item {La vecindad mide $\rho=(3,3)$, es un cuadro de $3\times 3$ símbolos}
  \item {La regla de evolución para el siguiente estado, asigna a la celda central el valor:}
  \begin{itemize}[<+->]
  \item ``viva'', si la celda central esta ``muerta'' y hay exactamente 3 vecinos vivos
  \item ``muerta'', si la celda central está ``viva'' y más de 3 (sobrepoblación) o menos de 2 (soledad) vecinos están vivos
  \item En cualquier otro caso, la celda mantiene su símbolo
  \end{itemize}
  \end{itemize}
  \begin{center}\includegraphics<+>[width=.6\textwidth]{vida1}\end{center}
\end{frame}

\begin{frame}[t]
\frametitle{Autómatas en 2D: Patrones de Turing}
\begin{columns}[t]
	\column{.7\textwidth}
	\begin{itemize}[<+->]
	\item Dos regiones elípticas, concéntricas a la celda central, cuya vida determinan
	\item Las celdas vivas en la elipse interna constituyen los activadores ($A$)
	\item Las celdas fuera de la elipse interna pero dentro de la externa constituyen los inhibidores ($I$)
	\item Hay un factor $w$ que dice que tan potentes son los inhibidores respecto a los activadores ($w=2$, significa que son el doble de potentes)
	\item Calcular $F = A - w * I$
	\begin{itemize}[<+->]
		\item Si $F >0$, la celda central vive
		\item Si $F<0$, la celda central muere
		\item Si $F=0$,  la celda central no cambia su valor
	\end{itemize}
	\end{itemize}
	\column{.3\textwidth}
	\begin{center}
		\includegraphics<1->[width=.9\textwidth]{fur1}
	\end{center}
\end{columns}
\end{frame}

\begin{frame}[t]
\frametitle{Autómatas en 2D: Patrones de Turing}
\begin{center}
	Programa ``Fur'' (biblioteca de modelos de Netlogo)
	
	\includegraphics<+>[width=.6\textwidth]{fur2}
\end{center}
\end{frame}

\begin{frame}[t]
\frametitle{Autómatas en 2D: Patrones de Turing: Observaciones}
\begin{itemize}[<+->]
	\item Nota como para ciertos valores de los parámetros, se observa la formación de patrones de ``manchas'', los cuales no dependen directamente de las reglas o el estado inicial de las celdas.
	\item Nota como en cierto momento estas ``manchas'', se estabilizan en cierta forma. Aunque algunas manchas, o fronteras de las mismas, podrían no llegar a estabilizarse.
	\item Nota también que esto aún puede considerarse autómata celular... discute por qué.
\end{itemize}

\end{frame}


\begin{frame}[t]
\frametitle{Percolación: definición}
\begin{itemize}[<+->]
	\item Se le llama percolación al traslado de alguna substancia a través de un medio ``poroso''. La substancia avanza siguiendo posiciones adyacentes de dicho medio.
	\item Ejemplos de esto son: en la ``cafetera de gotitas'', el café que atraviesa el filtro se dice que ha percolado. El petróleo u otros líquidos que se extraen de la tierra, pueden percolar a través de arena o rocas.
	\item Es relativamente sencilla su modelación mediante autómatas celulares...
	\item Ver programas ``Fire'' y ``Percolation'', en Netlogo.
\end{itemize}
\end{frame}

\begin{frame}[t]
\frametitle{Percolación: modelo}
\begin{itemize}[<+-| alert@+>]
	\item Una celda puede tener uno de tres estados: ``desocupada'', ``ocupada'' o ``utilizada''.
	\item El espacio disponible (rectángulo de celdas) se marcan unas celdas como desocupadas y otras como ocupadas, dependiendo de un valor llamado la ``densidad''.
	\item Luego de esto, en el llamado 1er estado, se marcan como ``utilizadas'', las celdas ocupadas de la columna más a la izquierda\footnote{percolación hacia la derecha, o el renglón de celdas ocupadas de más arriba, para percolar hacia abajo}.
	\item En el siguiente estado, se marcan como ``utilizadas'', las celdas del siguiente renglón que estén ocupadas y sean adyacentes\footnote{dentro de una vecindad de Moore, por ejemplo} a celdas utilizadas del estado anterior. Y así sucesivamente.
	\item Se dice que ``percola'' si al menos una celda de la columna de la derecha se torna ``utilizada''.
\end{itemize}
\end{frame}

\begin{frame}[t]
\frametitle{Percolación: variantes}
\begin{itemize}[<+->]
	\item Los modelos de Netlogo presentan unas variantes a esta definición
	\item En ``fire'', toda celda utilizada ``contagia'' a sus vecinas ocupadas, que pueden estar en columnas anteriores o siguientes.
	\item En ``percolation'', no hay celdas desocupadas, pero una utilizada ``contagia'' (o no) a cada una de sus dos vecinas del siguiente renglón, con cierta probabilidad, llamada ``porosidad''.
\end{itemize}

\end{frame}


\begin{frame}[t]
\frametitle{Percolación: ejemplos}
\begin{center}
\includegraphics<+>[height=.8\textheight]{fire}
\includegraphics<+>[height=.8\textheight]{percolation}
\end{center}
\end{frame}


\begin{frame}[t]
\frametitle{Percolación: obaservaciones}
\begin{itemize}[<+-| alert@+>]
	\item Variando la densidad, pudiera no ``percolar''. Dicho de otra forma, habrá un estado en que no haya suficientes celdas ocupadas adyacentes, a celdas utilizadas en el estado anterior.
	\item Se puede localizar el punto exacto\footnote{en realidad, como interviene el azar, esto depende de la precisión} de densidad en que ya se logra percolar\footnote{de nuevo, como interviene el azar, hay que realizar varios experimentos con la misma densidad, para ver si se obtiene el mismo resultado}.
	\item Observa que con densidad cero, no percola, y con densidad uno, si. Por lo que es posible encontrar el ``punto a partir del cual, el comportamiento cualitativo del sistema cambia''.
	\item Nota también que esto aún puede considerarse autómata celular... discute por qué.
\end{itemize}
\end{frame}

\section{Modelos Basados en Agentes}
\frame{\alert{Modelos Basados en Agentes}}

\begin{frame}[t]
\frametitle{Agentes}
\begin{itemize}[<+-| alert@+>]
	\item Agente: elemento individual autónomo de una simulación computacional
	\item Tiene:
	\begin{itemize}[<+-| alert@+>]
		\item Movimiento dentro de un ambiente (``mundo'' en Netlogo)
		\item Características como memoria (variables)
		\item La autonomía de acción (funciones) que se le dé
		\item Capacidad de interactuar con su ambiente y otros agentes
		\item Los estados son menos claros, pero puede inspeccionarse el estado del sistema en un momento dado
	\end{itemize}
	\item En general tienen menos restricciones que los autómatas
	\item ?`Un agente generaliza los autómatas? o ?`todo se puede modelar con autómatas?
\end{itemize}
\end{frame}

\begin{frame}[t]
\frametitle{Introducción a Netlogo}
\begin{itemize}[<+->]
	\item Lenguaje para Sistemas Complejos desarrollado por Uri Wilensky
	\item Puede verse como una generalización del lenguaje ``Logo'' (inicialmente desarrollado en el MIT)
	\item Permite manejar ``agentes'', que son principalmente las ``tortugas'' (que pueden adquirir personalidades), parches (estáticos, pero fuera de eso, son agentes también), y ``ligas'' (aristas entre tortugas, que permiten identificar las relacionadas, de ser necesario)
	\item Para programas chicos y medianos, su programación es bastante sencilla, para programas grandes, lo mejor es buscar otro lenguaje más avanzado (Python, Java, C, C++)
\end{itemize}

\end{frame}

\begin{frame}[t]
\frametitle{Elementos de Netlogo}
\begin{itemize}[<+->]
	\item Pestaña ``Ejecutar''
	\begin{itemize}[<+->]
	\item ``Añadir'' (``Botón'', ``Deslizador'', ``Interruptor'', ``Seleccionador'', ``Entrada'', ``Monitor'', ``Gráfico'', ``Salida'', ``Nota'')
	\item Velocidad
	\item ticks
	\item Actualizar de la Vista (``continuamente'', ``manualmente - ticks'')
	\item Configuración
	\item Mundo
	\item Terminal de Instrucciones, Borrar, observador>
	\end{itemize}
	\item Pestaña ``Información'' (Documentación del programa)
	\item Pestaña ``Código'' (Buscar, Comprobar, Procedimientos, Sangrado automático)
	\item Menús
\end{itemize}
\end{frame}

\begin{frame}[t]
\frametitle{Mundo}
\begin{itemize}[<+->]
	\item Es el ambiente donde se mueven los agentes
	\item Está compuesto de cuadritos contiguos llamados ``parcelas'' o ``parches''
	\item Consiste de agentes estáticos, ?`generalización de autómatas?
	\item Cada parche tiene coordenadas (enteras para su centro)
	\item En ``configuración'' (o ``editar'' con el botón derecho) se pueden modificar sus dimensiones, tamaño de parches, inicio y dirección de coordenadas, condiciones de frontera, etc.
	\item Se puede desplazar con el ratón, pero es único.
	\item El origen de sus coordenadas, el parche $(0,0)$, está en el centro.
	\item Las coordenadas $x$ y $y$ crecen hacia la derecha y hacia arriba, y decrecen en los otros sentidos.
\end{itemize}
\end{frame}


\begin{frame}[fragile]
\frametitle{Programas (códigos) iniciales}
\begin{itemize}[<+->]
	\item Por lo general en la ventana de código se agregan dos procedimientos: ``setup'' y ``go''
	\item ``setup'' ajusta las condiciones iniciales para el mundo y demás agentes
	\item ``go'' pone en marcha cualquier proceso que requiera la simulación
	\item Además de esto se suelen agregar deslizadores o cuadros de entrada para introducir o modificar los parámetros de la simulación
	\item El código mínimo para el setup es el siguiente:
\begin{Verbatim}[commandchars=\\\{\}]
\color{orange}to setup
    {\color{blue}clear-all}  ; se abrevia: ca
    ; agrega aquí inicialización de variables,
    ;  creación de tortugas, etc.
    ; nota los espacios a la izq. de estas líneas
    ;  indicando que están ''dentro'' de setup
    {\color{blue}reset-ticks}  ; inicializa el contador de pasos (ticks)
\color{orange}end
\end{Verbatim}
\end{itemize}
\end{frame}

\begin{frame}[t]
\frametitle{Programas iniciales}
\begin{itemize}[<+->]
	\item Luego de teclear este código, hay que hacer que se ejecute, por ejemplo con un botón cuyo único contenido es la llamada a la función ``setup'':
	\begin{center}{\includegraphics[width=.5\textwidth]{setup}}\end{center}
	\item Para agregar botones basta seleccionar ``Botón'' en el menú, presionar ``Añadir'' y dar click en alguna parte blanca junto al área del mundo (el ``mundo'' es el rectángulo negro de la pestaña ``Ejecutar'').
\end{itemize}
\end{frame}

\newcommand{\cb}{\color{blue}}
\newcommand{\cc}{\color{cyan}}
\newcommand{\cm}{\color{magenta}}
\newcommand{\cj}{\color{red}}

\begin{frame}[fragile]
\frametitle{Píntalo de rojo}
\begin{itemize}[<+->]
	\item Para que todos los parches realicen un mismo conjunto de acciones, utilizamos ``ask'': (después de teclear este código, agrega los botones necesarios)
	\begin{Verbatim}[commandchars=\\\{\}]
{\cc to} setup
  \cb ca
  \cb reset-ticks
\cc end

{\cc to} go
  {\cb ask} {\cm patches} [
    {\cb set} {\cm pcolor} \cj red
  ]
\cc end
\end{Verbatim}
\item ?`Qué pasa cuando presionas ``go'', pero antes haces que se ejecute ``mas lento''?, ?`Siempre es igual?
\end{itemize}
\end{frame}

\begin{frame}[t]
\frametitle{Tortugas borrachas}
\begin{itemize}[<+->]
	\item Veamos la creación y movimiento de tortugas
\includegraphics[width=.5\textwidth]{tortugas_borrachas}
\item Esto se puede leer: ``crea 10 tortugas, cada una: baja su pluma'', ``pídele a las tortugas: que apunten a un ángulo al azar en [0, 360); que avancen un paso'', al terminar con las tortugas: ``avanza el reloj''
\end{itemize}
\end{frame}

\begin{frame}[t]
\frametitle{Tortugas borrachas}
\begin{itemize}[<+->]
	\item Cada tortuga es un caminante al azar 2D. Haz las siguiente prueba:
	\item Cambia el tamaño y forma del mundo: 200x200, parcela de 1 pixel
	\item Aumenta la cantidad de tortugas a 10000
	\item Quítale el pen-down y haz que todas sean blancas
	\item Modifica el botón ``go'' para que trabaje ``continuamente''
	\item Cambia el menú de actualización a ``manualmente (ticks)''
	\item ?`Qué otra forma se te ocurre de generar caminantes al azar 2D?
\end{itemize}
\end{frame}

\begin{frame}[t]
\frametitle{Confeti}
\begin{itemize}[<+->]
	\item Ilustra: Movimiento de agentes
\end{itemize}

\end{frame}

\begin{frame}[t]
\frametitle{Pelotas}
\begin{itemize}[<+->]
	\item Ilustra: Interacción de los agentes y el medio
	\item Condiciones de frontera
\end{itemize}
\end{frame}

\begin{frame}[t]
\frametitle{presaDepredador}
\begin{itemize}[<+->]
	\item Ilustra: interacción de agentes entre si
	\item Presas: Deambulan aleatoriamente, cada cierto tiempo se reproducen
	\item Depredador: Al encontrar una presa cerca, se la come. Luego de cierto tiempo de no comer, muere. Luego de cierto tiempo, se reproduce.
	\item comportamiento complejo?
\end{itemize}
\end{frame}

\begin{frame}[t]
\frametitle{rapidosVSlentos}
\begin{itemize}[<+->]
	\item Ilustra: interacción de agentes con medio
	\item Interacción de agentes entre si
	\item discutir: emergencia de comportamiento complejo?
\end{itemize}
\end{frame}


\end{document}